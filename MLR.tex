% Options for packages loaded elsewhere
\PassOptionsToPackage{unicode}{hyperref}
\PassOptionsToPackage{hyphens}{url}
%
\documentclass[
]{article}
\usepackage{lmodern}
\usepackage{amssymb,amsmath}
\usepackage{ifxetex,ifluatex}
\ifnum 0\ifxetex 1\fi\ifluatex 1\fi=0 % if pdftex
  \usepackage[T1]{fontenc}
  \usepackage[utf8]{inputenc}
  \usepackage{textcomp} % provide euro and other symbols
\else % if luatex or xetex
  \usepackage{unicode-math}
  \defaultfontfeatures{Scale=MatchLowercase}
  \defaultfontfeatures[\rmfamily]{Ligatures=TeX,Scale=1}
\fi
% Use upquote if available, for straight quotes in verbatim environments
\IfFileExists{upquote.sty}{\usepackage{upquote}}{}
\IfFileExists{microtype.sty}{% use microtype if available
  \usepackage[]{microtype}
  \UseMicrotypeSet[protrusion]{basicmath} % disable protrusion for tt fonts
}{}
\makeatletter
\@ifundefined{KOMAClassName}{% if non-KOMA class
  \IfFileExists{parskip.sty}{%
    \usepackage{parskip}
  }{% else
    \setlength{\parindent}{0pt}
    \setlength{\parskip}{6pt plus 2pt minus 1pt}}
}{% if KOMA class
  \KOMAoptions{parskip=half}}
\makeatother
\usepackage{xcolor}
\IfFileExists{xurl.sty}{\usepackage{xurl}}{} % add URL line breaks if available
\IfFileExists{bookmark.sty}{\usepackage{bookmark}}{\usepackage{hyperref}}
\hypersetup{
  pdftitle={Multiple Linear Regression},
  pdfauthor={575 C1 Team 3},
  hidelinks,
  pdfcreator={LaTeX via pandoc}}
\urlstyle{same} % disable monospaced font for URLs
\usepackage[margin=1in]{geometry}
\usepackage{color}
\usepackage{fancyvrb}
\newcommand{\VerbBar}{|}
\newcommand{\VERB}{\Verb[commandchars=\\\{\}]}
\DefineVerbatimEnvironment{Highlighting}{Verbatim}{commandchars=\\\{\}}
% Add ',fontsize=\small' for more characters per line
\usepackage{framed}
\definecolor{shadecolor}{RGB}{248,248,248}
\newenvironment{Shaded}{\begin{snugshade}}{\end{snugshade}}
\newcommand{\AlertTok}[1]{\textcolor[rgb]{0.94,0.16,0.16}{#1}}
\newcommand{\AnnotationTok}[1]{\textcolor[rgb]{0.56,0.35,0.01}{\textbf{\textit{#1}}}}
\newcommand{\AttributeTok}[1]{\textcolor[rgb]{0.77,0.63,0.00}{#1}}
\newcommand{\BaseNTok}[1]{\textcolor[rgb]{0.00,0.00,0.81}{#1}}
\newcommand{\BuiltInTok}[1]{#1}
\newcommand{\CharTok}[1]{\textcolor[rgb]{0.31,0.60,0.02}{#1}}
\newcommand{\CommentTok}[1]{\textcolor[rgb]{0.56,0.35,0.01}{\textit{#1}}}
\newcommand{\CommentVarTok}[1]{\textcolor[rgb]{0.56,0.35,0.01}{\textbf{\textit{#1}}}}
\newcommand{\ConstantTok}[1]{\textcolor[rgb]{0.00,0.00,0.00}{#1}}
\newcommand{\ControlFlowTok}[1]{\textcolor[rgb]{0.13,0.29,0.53}{\textbf{#1}}}
\newcommand{\DataTypeTok}[1]{\textcolor[rgb]{0.13,0.29,0.53}{#1}}
\newcommand{\DecValTok}[1]{\textcolor[rgb]{0.00,0.00,0.81}{#1}}
\newcommand{\DocumentationTok}[1]{\textcolor[rgb]{0.56,0.35,0.01}{\textbf{\textit{#1}}}}
\newcommand{\ErrorTok}[1]{\textcolor[rgb]{0.64,0.00,0.00}{\textbf{#1}}}
\newcommand{\ExtensionTok}[1]{#1}
\newcommand{\FloatTok}[1]{\textcolor[rgb]{0.00,0.00,0.81}{#1}}
\newcommand{\FunctionTok}[1]{\textcolor[rgb]{0.00,0.00,0.00}{#1}}
\newcommand{\ImportTok}[1]{#1}
\newcommand{\InformationTok}[1]{\textcolor[rgb]{0.56,0.35,0.01}{\textbf{\textit{#1}}}}
\newcommand{\KeywordTok}[1]{\textcolor[rgb]{0.13,0.29,0.53}{\textbf{#1}}}
\newcommand{\NormalTok}[1]{#1}
\newcommand{\OperatorTok}[1]{\textcolor[rgb]{0.81,0.36,0.00}{\textbf{#1}}}
\newcommand{\OtherTok}[1]{\textcolor[rgb]{0.56,0.35,0.01}{#1}}
\newcommand{\PreprocessorTok}[1]{\textcolor[rgb]{0.56,0.35,0.01}{\textit{#1}}}
\newcommand{\RegionMarkerTok}[1]{#1}
\newcommand{\SpecialCharTok}[1]{\textcolor[rgb]{0.00,0.00,0.00}{#1}}
\newcommand{\SpecialStringTok}[1]{\textcolor[rgb]{0.31,0.60,0.02}{#1}}
\newcommand{\StringTok}[1]{\textcolor[rgb]{0.31,0.60,0.02}{#1}}
\newcommand{\VariableTok}[1]{\textcolor[rgb]{0.00,0.00,0.00}{#1}}
\newcommand{\VerbatimStringTok}[1]{\textcolor[rgb]{0.31,0.60,0.02}{#1}}
\newcommand{\WarningTok}[1]{\textcolor[rgb]{0.56,0.35,0.01}{\textbf{\textit{#1}}}}
\usepackage{graphicx,grffile}
\makeatletter
\def\maxwidth{\ifdim\Gin@nat@width>\linewidth\linewidth\else\Gin@nat@width\fi}
\def\maxheight{\ifdim\Gin@nat@height>\textheight\textheight\else\Gin@nat@height\fi}
\makeatother
% Scale images if necessary, so that they will not overflow the page
% margins by default, and it is still possible to overwrite the defaults
% using explicit options in \includegraphics[width, height, ...]{}
\setkeys{Gin}{width=\maxwidth,height=\maxheight,keepaspectratio}
% Set default figure placement to htbp
\makeatletter
\def\fps@figure{htbp}
\makeatother
\setlength{\emergencystretch}{3em} % prevent overfull lines
\providecommand{\tightlist}{%
  \setlength{\itemsep}{0pt}\setlength{\parskip}{0pt}}
\setcounter{secnumdepth}{-\maxdimen} % remove section numbering

\title{Multiple Linear Regression}
\author{575 C1 Team 3}
\date{2020/10/26}

\begin{document}
\maketitle

Team Members Dingjie Chen, Siwen He, Hanzi Yu, Jiaqi Yin, Runsheng Wang

In this deliverable, we are performing Multiple Linear Regression on the
facebook dataset. We first load the packages needed to perform the
analysis and read in the delimited file. We modified the column names of
the CSV file so that column names would not contain space, as space is
not a valid name character in ggplot. We used the complete.cases()
function to handle NA values. Also note that ``Category'' and ``Paid''
variables are being interpreted as a double by the col\_guess()
function. To use these two variables as categorical, we could call the
as.factor() function.

\begin{Shaded}
\begin{Highlighting}[]
\CommentTok{# loading packages}
\KeywordTok{suppressPackageStartupMessages}\NormalTok{(}\KeywordTok{library}\NormalTok{(tidyverse))}
\KeywordTok{suppressPackageStartupMessages}\NormalTok{(}\KeywordTok{library}\NormalTok{(modelr))}

\CommentTok{# loading datasets}
\NormalTok{fb <-}\StringTok{ }\KeywordTok{read_delim}\NormalTok{(}\StringTok{"dataset_Facebook.csv"}\NormalTok{, }\DataTypeTok{delim =} \StringTok{";"}\NormalTok{)}
\NormalTok{fb <-}\StringTok{ }\NormalTok{fb[}\KeywordTok{complete.cases}\NormalTok{(fb), ]}

\CommentTok{# center titles for ggplot}
\KeywordTok{theme_update}\NormalTok{(}\DataTypeTok{plot.title =} \KeywordTok{element_text}\NormalTok{(}\DataTypeTok{hjust =} \FloatTok{0.5}\NormalTok{))}
\end{Highlighting}
\end{Shaded}

To identify potentially meaningful relationships, we construct a heatmap
for T\_Impression and T\_Interactions. Since facebook does a horrible
job at explaining their metrics, it makes sense for us to first define
clearly what each variable is accounting for. After doing some research
online, we found that Facebook calculates

\begin{Shaded}
\begin{Highlighting}[]
\CommentTok{# construct heatmap}
\NormalTok{fb }\OperatorTok\StringTok{ }\KeywordTok{transmute}\NormalTok{(}\DataTypeTok{T_Impression =} \KeywordTok{cut_number}\NormalTok{(T_Impression, }\DecValTok{6}\NormalTok{), }\DataTypeTok{T_Interactions =} \KeywordTok{cut_number}\NormalTok{(T_Interactions, }\DecValTok{6}\NormalTok{)) }\OperatorTok\StringTok{ }\KeywordTok{count}\NormalTok{(T_Impression, T_Interactions) }\OperatorTok\StringTok{ }\KeywordTok{ggplot}\NormalTok{(}\KeywordTok{aes}\NormalTok{(T_Impression, T_Interactions)) }\OperatorTok{+}\StringTok{ }\KeywordTok{geom_tile}\NormalTok{(}\KeywordTok{aes}\NormalTok{(}\DataTypeTok{fill =}\NormalTok{ n)) }\OperatorTok{+}\StringTok{ }\KeywordTok{scale_x_discrete}\NormalTok{(}\DataTypeTok{labels =}\NormalTok{ abbreviate) }\OperatorTok{+}\StringTok{ }\KeywordTok{scale_fill_viridis_c}\NormalTok{()}
\end{Highlighting}
\end{Shaded}

\includegraphics{MLR_files/figure-latex/heatmap-1.pdf}

On the heatmap, lighter colors correspond to higher number of posts that
belongs to the bin with specific T\_Impression and T\_Interactions. We
see that lighter colors are clustering along the diagonal of the
heatmap, indicating a potentially positive correlation between the two
variables. This heatmap motivate us to plot T\_Interactions against
T\_Impression, as illustrated below.

First, we identify outliers and remove them.

\begin{Shaded}
\begin{Highlighting}[]
\CommentTok{# clean outliers}
\NormalTok{fbmean =}\StringTok{ }\KeywordTok{mean}\NormalTok{(fb}\OperatorTok{$}\NormalTok{T_Impression)}
\NormalTok{fbsd =}\StringTok{ }\KeywordTok{sd}\NormalTok{(fb}\OperatorTok{$}\NormalTok{T_Impression)}
\NormalTok{fb.clean <-}\StringTok{ }\NormalTok{fb }\OperatorTok\StringTok{ }\KeywordTok{filter}\NormalTok{(T_Impression }\OperatorTok{<=}\StringTok{ }\NormalTok{fbmean}\OperatorTok{+}\DecValTok{3}\OperatorTok{*}\NormalTok{fbsd)}
\CommentTok{# calculate percentage of datapoints considered}
\NormalTok{removedt <-}\StringTok{ }\NormalTok{(}\DecValTok{1} \OperatorTok{-}\StringTok{ }\KeywordTok{nrow}\NormalTok{(fb.clean)}\OperatorTok{/}\KeywordTok{nrow}\NormalTok{(fb))}\OperatorTok{*}\DecValTok{100}
\NormalTok{percentage_tibble <-}\StringTok{ }\KeywordTok{tribble}\NormalTok{(}\OperatorTok{~}\NormalTok{Variable, }\OperatorTok{~}\NormalTok{Percentage_Removed, }\StringTok{"Overall"}\NormalTok{, removedt)}
\NormalTok{(percentage_tibble)}
\end{Highlighting}
\end{Shaded}

\begin{verbatim}
## # A tibble: 1 x 2
##   Variable Percentage_Removed
##   <chr>                 <dbl>
## 1 Overall                1.41
\end{verbatim}

Notice that by including datapoints within 3 standard deviation from the
mean removed 1.14\% of the total data points.

Now, we proceed to calculate OLS

\begin{Shaded}
\begin{Highlighting}[]
\CommentTok{# calculate OLS}
\NormalTok{m.ols <-}\StringTok{ }\KeywordTok{lm}\NormalTok{(T_Interactions }\OperatorTok{~}\StringTok{ }\NormalTok{T_Impression, }\DataTypeTok{data =}\NormalTok{ fb.clean)}
\KeywordTok{summary}\NormalTok{(m.ols)}
\end{Highlighting}
\end{Shaded}

\begin{verbatim}
## 
## Call:
## lm(formula = T_Interactions ~ T_Impression, data = fb.clean)
## 
## Residuals:
##    Min     1Q Median     3Q    Max 
## -835.1  -92.1  -43.1   39.7 1624.1 
## 
## Coefficients:
##               Estimate Std. Error t value Pr(>|t|)    
## (Intercept)  1.229e+02  1.211e+01   10.15   <2e-16 ***
## T_Impression 3.323e-03  2.879e-04   11.54   <2e-16 ***
## ---
## Signif. codes:  0 '***' 0.001 '**' 0.01 '*' 0.05 '.' 0.1 ' ' 1
## 
## Residual standard error: 225.9 on 486 degrees of freedom
## Multiple R-squared:  0.2152, Adjusted R-squared:  0.2136 
## F-statistic: 133.3 on 1 and 486 DF,  p-value: < 2.2e-16
\end{verbatim}

\begin{Shaded}
\begin{Highlighting}[]
\KeywordTok{round}\NormalTok{(}\KeywordTok{confint}\NormalTok{(m.ols,}\DataTypeTok{level=}\FloatTok{0.95}\NormalTok{),}\DecValTok{6}\NormalTok{)}
\end{Highlighting}
\end{Shaded}

\begin{verbatim}
##                  2.5 %     97.5 %
## (Intercept)  99.149803 146.746087
## T_Impression  0.002758   0.003889
\end{verbatim}

\begin{Shaded}
\begin{Highlighting}[]
\KeywordTok{vcov}\NormalTok{(m.ols)}
\end{Highlighting}
\end{Shaded}

\begin{verbatim}
##                (Intercept)  T_Impression
## (Intercept)  146.697977566 -1.867812e-03
## T_Impression  -0.001867812  8.287600e-08
\end{verbatim}

As can be seen from the output, the fit is not quite good, with an
adjusted R-square value of 0.1154. The intercept and slope seem to have
a high t value and a low p value, indicating that T\_Impression is a
significant predictor. However, the relationship under the linear model
is not a strong one. The confidence interval for the intercept is very
wide, whereas the confidence interval on slope is very narrow. This
outcome agrees with our previous analysis that the relationship between
T\_Impression and T\_Interactions might not be linear. We will plot the
OLS model below with residuals, and offer an alternative solution to
fitting this data later in this document.

\begin{Shaded}
\begin{Highlighting}[]
\CommentTok{#plot OLS with predictions}
\KeywordTok{ggplot}\NormalTok{(fb.clean }\OperatorTok\StringTok{ }\KeywordTok{add_predictions}\NormalTok{(m.ols), }\KeywordTok{aes}\NormalTok{(T_Impression, T_Interactions)) }\OperatorTok{+}\StringTok{ }\KeywordTok{geom_point}\NormalTok{(}\DataTypeTok{size =} \FloatTok{0.1}\NormalTok{) }\OperatorTok{+}\StringTok{ }\KeywordTok{geom_line}\NormalTok{(}\KeywordTok{aes}\NormalTok{(}\DataTypeTok{y=}\NormalTok{pred), }\DataTypeTok{color =} \StringTok{"blue"}\NormalTok{)}
\end{Highlighting}
\end{Shaded}

\includegraphics{MLR_files/figure-latex/OLS_plot-1.pdf}

\begin{Shaded}
\begin{Highlighting}[]
\CommentTok{# plot residual}
\KeywordTok{ggplot}\NormalTok{(fb.clean }\OperatorTok\StringTok{ }\KeywordTok{add_residuals}\NormalTok{(m.ols), }\KeywordTok{aes}\NormalTok{(T_Impression, resid))}\OperatorTok{+}\KeywordTok{geom_hex}\NormalTok{(}\DataTypeTok{alpha =} \FloatTok{0.7}\NormalTok{)}\OperatorTok{+}\KeywordTok{geom_hline}\NormalTok{(}\KeywordTok{aes}\NormalTok{(}\DataTypeTok{yintercept =} \DecValTok{0}\NormalTok{))}\OperatorTok{+}\KeywordTok{scale_fill_viridis_c}\NormalTok{()}
\end{Highlighting}
\end{Shaded}

\includegraphics{MLR_files/figure-latex/OLS_plot-2.pdf}

As seen in the graphs above, the linear model doesn't fit the data quite
well. The residual plot also indicate that we have the assumption of
equal variances do not hold in this case, as residuals tend to be
greater for higher values of T\_Impression. Notice that data seem to
cluster below T\_Impression = 30000. We perform OLS again on
T\_Impression \textless= 30000

\begin{Shaded}
\begin{Highlighting}[]
\CommentTok{# clean outliers}
\NormalTok{fb.clean1 <-}\StringTok{ }\NormalTok{fb }\OperatorTok\StringTok{ }\KeywordTok{filter}\NormalTok{(T_Impression }\OperatorTok{<=}\StringTok{ }\DecValTok{30000}\NormalTok{)}
\CommentTok{# calculate percentage of datapoints considered}
\NormalTok{removedt1 <-}\StringTok{ }\NormalTok{(}\DecValTok{1} \OperatorTok{-}\StringTok{ }\KeywordTok{nrow}\NormalTok{(fb.clean1)}\OperatorTok{/}\KeywordTok{nrow}\NormalTok{(fb))}\OperatorTok{*}\DecValTok{100}
\NormalTok{percentage_tibble <-}\StringTok{ }\KeywordTok{tribble}\NormalTok{(}\OperatorTok{~}\NormalTok{Variable, }\OperatorTok{~}\NormalTok{Percentage_Removed, }\StringTok{"Overall"}\NormalTok{, removedt1)}
\NormalTok{(percentage_tibble)}
\end{Highlighting}
\end{Shaded}

\begin{verbatim}
## # A tibble: 1 x 2
##   Variable Percentage_Removed
##   <chr>                 <dbl>
## 1 Overall                21.0
\end{verbatim}

\begin{Shaded}
\begin{Highlighting}[]
\CommentTok{# calculate OLS}
\NormalTok{m.ols1 <-}\StringTok{ }\KeywordTok{lm}\NormalTok{(T_Interactions }\OperatorTok{~}\StringTok{ }\NormalTok{T_Impression, }\DataTypeTok{data =}\NormalTok{ fb.clean1)}
\KeywordTok{summary}\NormalTok{(m.ols1)}
\end{Highlighting}
\end{Shaded}

\begin{verbatim}
## 
## Call:
## lm(formula = T_Interactions ~ T_Impression, data = fb.clean1)
## 
## Residuals:
##     Min      1Q  Median      3Q     Max 
## -227.86  -54.89  -15.76   43.92  482.86 
## 
## Coefficients:
##               Estimate Std. Error t value Pr(>|t|)    
## (Intercept)  5.187e+01  8.566e+00   6.055  3.3e-09 ***
## T_Impression 8.961e-03  7.695e-04  11.646  < 2e-16 ***
## ---
## Signif. codes:  0 '***' 0.001 '**' 0.01 '*' 0.05 '.' 0.1 ' ' 1
## 
## Residual standard error: 93.64 on 389 degrees of freedom
## Multiple R-squared:  0.2585, Adjusted R-squared:  0.2566 
## F-statistic: 135.6 on 1 and 389 DF,  p-value: < 2.2e-16
\end{verbatim}

\begin{Shaded}
\begin{Highlighting}[]
\KeywordTok{round}\NormalTok{(}\KeywordTok{confint}\NormalTok{(m.ols1,}\DataTypeTok{level=}\FloatTok{0.95}\NormalTok{),}\DecValTok{6}\NormalTok{)}
\end{Highlighting}
\end{Shaded}

\begin{verbatim}
##                  2.5 %    97.5 %
## (Intercept)  35.029691 68.713121
## T_Impression  0.007448  0.010474
\end{verbatim}

\begin{Shaded}
\begin{Highlighting}[]
\KeywordTok{vcov}\NormalTok{(m.ols1)}
\end{Highlighting}
\end{Shaded}

\begin{verbatim}
##              (Intercept)  T_Impression
## (Intercept)  73.37866365 -5.492540e-03
## T_Impression -0.00549254  5.920583e-07
\end{verbatim}

\begin{Shaded}
\begin{Highlighting}[]
\CommentTok{#plot OLS with predictions}
\KeywordTok{ggplot}\NormalTok{(fb.clean1 }\OperatorTok\StringTok{ }\KeywordTok{add_predictions}\NormalTok{(m.ols1), }\KeywordTok{aes}\NormalTok{(T_Impression, T_Interactions)) }\OperatorTok{+}\StringTok{ }\KeywordTok{geom_point}\NormalTok{(}\DataTypeTok{size =} \FloatTok{0.1}\NormalTok{) }\OperatorTok{+}\StringTok{ }\KeywordTok{geom_line}\NormalTok{(}\KeywordTok{aes}\NormalTok{(}\DataTypeTok{y=}\NormalTok{pred), }\DataTypeTok{color =} \StringTok{"blue"}\NormalTok{)}
\end{Highlighting}
\end{Shaded}

\includegraphics{MLR_files/figure-latex/30k-1.pdf}

\begin{Shaded}
\begin{Highlighting}[]
\CommentTok{# plot residual}
\KeywordTok{ggplot}\NormalTok{(fb.clean1 }\OperatorTok\StringTok{ }\KeywordTok{add_residuals}\NormalTok{(m.ols1), }\KeywordTok{aes}\NormalTok{(T_Impression, resid))}\OperatorTok{+}\KeywordTok{geom_hex}\NormalTok{(}\DataTypeTok{alpha =} \FloatTok{0.7}\NormalTok{)}\OperatorTok{+}\KeywordTok{geom_hline}\NormalTok{(}\KeywordTok{aes}\NormalTok{(}\DataTypeTok{yintercept =} \DecValTok{0}\NormalTok{))}\OperatorTok{+}\KeywordTok{scale_fill_viridis_c}\NormalTok{()}
\end{Highlighting}
\end{Shaded}

\includegraphics{MLR_files/figure-latex/30k-2.pdf}

The output still has significant values for parameter estimation with
slightly improved Adjusted R-Square. The output of this plot and the
visual trend of the scatterplot motivates us to do a log transformation
on the variable.

\begin{Shaded}
\begin{Highlighting}[]
\CommentTok{# log transform}
\NormalTok{logfb <-}\StringTok{ }\NormalTok{fb.clean }\OperatorTok\StringTok{ }\KeywordTok{transmute}\NormalTok{(}\DataTypeTok{L_T_Interactions =} \KeywordTok{log}\NormalTok{(T_Interactions), }\DataTypeTok{L_T_Impression =} \KeywordTok{log}\NormalTok{(T_Impression)) }\OperatorTok\StringTok{ }\KeywordTok{filter}\NormalTok{(}\OperatorTok{!}\KeywordTok{is.infinite}\NormalTok{(L_T_Interactions))}
\NormalTok{m.log <-}\StringTok{ }\KeywordTok{lm}\NormalTok{(L_T_Interactions }\OperatorTok{~}\StringTok{ }\NormalTok{L_T_Impression, }\DataTypeTok{data =}\NormalTok{ logfb)}
\KeywordTok{summary}\NormalTok{(m.log)}
\end{Highlighting}
\end{Shaded}

\begin{verbatim}
## 
## Call:
## lm(formula = L_T_Interactions ~ L_T_Impression, data = logfb)
## 
## Residuals:
##     Min      1Q  Median      3Q     Max 
## -3.9347 -0.4297  0.1383  0.5774  1.7011 
## 
## Coefficients:
##                Estimate Std. Error t value Pr(>|t|)    
## (Intercept)    -1.19130    0.33136  -3.595 0.000358 ***
## L_T_Impression  0.63860    0.03517  18.160  < 2e-16 ***
## ---
## Signif. codes:  0 '***' 0.001 '**' 0.01 '*' 0.05 '.' 0.1 ' ' 1
## 
## Residual standard error: 0.8358 on 481 degrees of freedom
## Multiple R-squared:  0.4067, Adjusted R-squared:  0.4055 
## F-statistic: 329.8 on 1 and 481 DF,  p-value: < 2.2e-16
\end{verbatim}

This model is considerably better than the previous one, with very low p
values for both parameters of interest. This output indicates that the
both beta0 and beta1 are significant. Adjusted R-squared value is also
high, at 0.3989, indicating that this fit is better compared to the
linear fit. The graphs for the log-transformed variables are given
below.

\begin{Shaded}
\begin{Highlighting}[]
\CommentTok{# plot log with predictions}
\KeywordTok{ggplot}\NormalTok{(logfb }\OperatorTok\StringTok{ }\KeywordTok{add_predictions}\NormalTok{(m.log), }\KeywordTok{aes}\NormalTok{(L_T_Impression, L_T_Interactions)) }\OperatorTok{+}\StringTok{ }\KeywordTok{geom_point}\NormalTok{(}\DataTypeTok{size =} \FloatTok{0.1}\NormalTok{) }\OperatorTok{+}\StringTok{ }\KeywordTok{geom_line}\NormalTok{(}\KeywordTok{aes}\NormalTok{(}\DataTypeTok{y=}\NormalTok{pred), }\DataTypeTok{color =} \StringTok{"blue"}\NormalTok{)}
\end{Highlighting}
\end{Shaded}

\includegraphics{MLR_files/figure-latex/LOG_plot-1.pdf}

\begin{Shaded}
\begin{Highlighting}[]
\CommentTok{# plot residual}
\KeywordTok{ggplot}\NormalTok{(logfb }\OperatorTok\StringTok{ }\KeywordTok{add_residuals}\NormalTok{(m.log), }\KeywordTok{aes}\NormalTok{(L_T_Interactions, resid))}\OperatorTok{+}\KeywordTok{geom_hex}\NormalTok{(}\DataTypeTok{alpha =} \FloatTok{0.7}\NormalTok{)}\OperatorTok{+}\KeywordTok{geom_hline}\NormalTok{(}\KeywordTok{aes}\NormalTok{(}\DataTypeTok{yintercept =} \DecValTok{0}\NormalTok{))}\OperatorTok{+}\KeywordTok{scale_fill_viridis_c}\NormalTok{()}
\end{Highlighting}
\end{Shaded}

\includegraphics{MLR_files/figure-latex/LOG_plot-2.pdf}

Notice that the log plot follows a roughly linear trend, with residuals
roughly clustering around the y=0 line. There seems to be a linear trend
in the residual, indicating that the linearity assumption could be
violated. If log transformation yields a considerably good linear fit,
the non-transformed data could exhibit an exponential relationship. We
won't expand on this concept for this project deliverable, but it will
be considered when we construct our final project.

\end{document}
